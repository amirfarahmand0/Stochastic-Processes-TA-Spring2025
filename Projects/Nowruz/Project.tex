\documentclass[12pt]{article}
\usepackage{amssymb}
\usepackage{amsmath}
\usepackage{makeidx}

\ExplSyntaxOn
\cs_set_eq:NN \etex_iffontchar:D \tex_iffontchar:D
\ExplSyntaxOff
\usepackage{mathtools}
\usepackage{hyperref}
\hypersetup{
	colorlinks=true,
	linkcolor=[rgb]{0,0,0.6953125},
	filecolor=[rgb]{0,0,0.6953125},      
	urlcolor=[rgb]{0,0,0.6953125},
	citecolor=[rgb]{0,0,0.6953125}
}
\newcommand{\hint}[1]{
	\begin{flushleft}
		\hyperlink{#1}{راهنمایی}
	\end{flushleft}
}
\usepackage{UTProblemSet}

\begin{document}
			%%%%%%%%%%%%%%%%%%% سربرگ سند %%%%%%%%%%%%%%%%%%%
	\pagestyle{empty}
	\heading
	{
		فرآیندهای تصادفی
	}
	{
		دکتر صفری
	}
	{زمستان 1403}
	{16 فروردین}
	{
		پروژه نوروزی
	}
	\inspiringQuotation
	{Probability theory is nothing but common sense reduced to calculation. }
	{Pierre-Simon Laplace (1749-1827)}
	
			%%%%%%%%%%%%%%%%%%% بخش سوالات %%%%%%%%%%%%%%%%%%%


    \section{
        فرآیند تصمیم مارکف متناهی
    }   
    \subsection{فرآیند پاداش مارکف}
    یک فرآیند تصادفی مارکف \(\{ X_n \}\) با تابع انتقال \(P(X_n = i | X_{n-1} = j) = p(i,j)\) را در نظر بگیرید. متناظر با هر وضعیت \(X_t\) که فرآیند در آن قرار دارد یک پاداش \(R_t\) (تصادفی یا قطعی) تعلق می گیرد.
    \[R_t = R(X_t) \quad \textrm{یا} \quad P(R_t= r | X_t=x) = p_r(x) \quad , \quad r_t(x) = E[R_t | X_t = x] \]
    \paragraph{تعریف.} به چهارتایی \(<S , P , R , \gamma>\) یک فرآیند پاداش مارکف\LTRfootnote{Markov Reward Process} می گوییم. 
    \begin{description}
        \item[\(S\)] مجموعه حالاتی که فرآیند در آن قرار می گیرد
        \item[\(P\)] ماتریس/تابع احتمال انتقال
        \item[\(R\)] توزیع پاداش یا مقدار پاداش در هر حالات
        \item[\(\gamma\)] نرخ کاهش اثر پاداش قدیمی
    \end{description}
    \begin{itemize}
        \item مجموع پاداش تا لحظه \(t\) \[ G_t = \sum_{n=0}^{t} \gamma^{t - n} R_n\]
    \end{itemize}
\end{document}