\documentclass[11pt, a4, twoside]{article}

\usepackage{
    amsmath,
    amsfonts,
    amssymb
}

\usepackage{subcaption}
\usepackage{xcolor}
\usepackage{
	tikz,
	pgfplots
}


\ExplSyntaxOn
\cs_set_eq:NN \etex_iffontchar:D \tex_iffontchar:D
\ExplSyntaxOff

\usepackage{mathtools}
\usepackage{hyperref}
\hypersetup{
	colorlinks=true,
	linkcolor=[rgb]{0,0,0.6953125},
	filecolor=[rgb]{0,0,0.6953125},      
	urlcolor=[rgb]{0,0,0.6953125},
	citecolor=[rgb]{0,0,0.6953125}
}
\newcommand{\hint}[1]{
	\begin{flushleft}
		\hyperlink{#1}{راهنمایی}
	\end{flushleft}
}
\usepackage{UTProblemSet}
\DefaultMathDigits

\begin{document}
			%%%%%%%%%%%%%%%%%%% سربرگ سند %%%%%%%%%%%%%%%%%%%
	\pagestyle{empty}
	\heading
	{فرآیندهای تصادفی}
	{دکتر صفری}
	{نیمسال دوم 1403-1404}
    {سه‌شنبه، 16 اردیبهشت 23:59}
    {تمرین سری هفتم}
	\inspiringQuotation
	{\\We have to remember that what we observe is not nature in itself but nature exposed to our method of questioning.}
	{Werner Heisenberg (1901-1976)}
			%%%%%%%%%%%%%%%%%%% بخش سوالات %%%%%%%%%%%%%%%%%%%
	\begin{problem}
		خرابی در طول یک رشته طناب با نرخ $\lambda = 2$ در هر فوت رخ می‌دهند.

		\begin{itemize}
		\item[(الف)] احتمال اینکه در اولین فوت رشته طناب هیچ نقصی وجود نداشته باشد را محاسبه کنید.
		\item[(ب)] احتمال شرطی اینکه در دومین فوت رشته طناب هیچ نقصی وجود نداشته باشد، با فرض اینکه در اولین فوت فقط یک نقص وجود داشته است، را محاسبه کنید.
		\end{itemize}
	\end{problem}

	\begin{problem}			
		فرض کنید $p_k = \Pr\{X = k\}$ تابع جرم احتمال مربوط به توزیع پواسون با پارامتر $\lambda$ باشد. نشان دهید که:
		\(
		p_0 = \exp\{-\lambda\}
		\)
		و اینکه $p_k$ به‌صورت بازگشتی توسط رابطه‌ی زیر محاسبه می‌شود:
		\(
		p_k = (\lambda / k) p_{k-1}
		\)
	\end{problem}
	\begin{problem}
		فرض کنید $X$ و $Y$ دو متغیر تصادفی مستقل با توزیع پواسون و پارامترهای $\alpha$ و $\beta$ به ترتیب باشند. توزیع شرطی $X$ را با فرض اینکه $N = X + Y = n$ تعیین کنید.
	\end{problem}

	\begin{problem}
		یک فروشگاه ساعت ۸ صبح باز می‌شود. از ساعت ۸ تا ۱۰ صبح، مشتریان با نرخ پواسون برابر با ۴ نفر در ساعت وارد فروشگاه می‌شوند. بین ساعت ۱۰ صبح تا ۱۲ ظهر، نرخ ورود مشتریان برابر با ۸ نفر در ساعت است. از ساعت ۱۲ ظهر تا ۲ بعدازظهر، نرخ ورود مشتریان به‌صورت پیوسته از ۸ نفر در ساعت (در ۱۲ ظهر) تا ۱۰ نفر در ساعت (در ۲ بعدازظهر) افزایش می‌یابد؛ و از ساعت ۲ تا ۵ بعدازظهر، نرخ ورود مشتریان به‌صورت پیوسته از ۱۰ نفر در ساعت (در ۲ بعدازظهر) تا ۴ نفر در ساعت (در ۵ بعدازظهر) کاهش می‌یابد. 

		توزیع احتمالاتی تعداد مشتریانی که در یک روز مشخص وارد فروشگاه می‌شوند را تعیین کنید
	\end{problem}
\end{document}
