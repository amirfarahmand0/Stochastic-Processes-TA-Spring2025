\documentclass[11pt, a4, twoside]{article}

\usepackage{
    amsmath,
    amsfonts,
    amssymb
}

\usepackage{subcaption}
\usepackage{xcolor}
\usepackage{
	tikz,
	pgfplots
}


\ExplSyntaxOn
\cs_set_eq:NN \etex_iffontchar:D \tex_iffontchar:D
\ExplSyntaxOff

\usepackage{mathtools}
\usepackage{hyperref}
\hypersetup{
	colorlinks=true,
	linkcolor=[rgb]{0,0,0.6953125},
	filecolor=[rgb]{0,0,0.6953125},      
	urlcolor=[rgb]{0,0,0.6953125},
	citecolor=[rgb]{0,0,0.6953125}
}
\newcommand{\hint}[1]{
	\begin{flushleft}
		\hyperlink{#1}{راهنمایی}
	\end{flushleft}
}
\usepackage{UTProblemSet}
\DefaultMathDigits

\begin{document}
			%%%%%%%%%%%%%%%%%%% سربرگ سند %%%%%%%%%%%%%%%%%%%
	\pagestyle{empty}
	\heading
	{فرآیندهای تصادفی}
	{دکتر صفری}
	{نیمسال دوم 1403-1404}
    {جمعه، 17 مرداد 23:59}
    {تمرین سری سیزدهم}
	\inspiringQuotation
{
    In mathematics, you don't understand things. You just get used to them.
}
{John von Neumann (1903–1957)}
			%%%%%%%%%%%%%%%%%%% بخش سوالات %%%%%%%%%%%%%%%%%%%
		\begin{problem}
			فرض کنید \( X(t) \) یک فرآیند پواسن همگن با پارامتر \( \lambda \) باشد. کوواریانس بین \( X(t) \) و \( X(t+\tau) \) را برای \( t>0 \) و \( \tau>0 \) محاسبه کنید، یعنی عبارت زیر را به‌دست آورید:
			\[
				\text{Cov}\big(X(t), X(t+\tau)\big) = E\left[\big(X(t) - E[X(t)]\big)\big(X(t+\tau) - E[X(t+\tau)]\big)\right].
			\]
		\end{problem}

		\begin{problem}
						
			مشاهده یک اتم رادیواکتیو را از زمان $t=0$ آغاز می‌کنیم. این اتم در زمان $t>0$ دچار واپاشی شده و از حالت رادیواکتیو خارج می‌شود. توزیع زمانی این واپاشی به صورت زیر است:

			\[
			F(\tau) = 
			\begin{cases} 
			0, & \tau < 0 \\
			1 - e^{-\lambda t}, & \tau \geq 0 
			\end{cases}
			= \Pr\{t \leq \tau\}.
			\]

			حالت اتم در زمان $t$ را به عنوان یک متغیر تصادفی در نظر می‌گیریم:

			\[
			x_t = 
			\begin{cases} 
			0 & \text{اگر اتم در زمان $t$ رادیواکتیو باشد} \\
			1 & \text{اگر اتم در زمان $t$ رادیواکتیو نباشد}
			\end{cases}
			\]

			مجموعه متغیرهای $\{x_t\}$ یک فرآیند تصادفی را تعریف می‌کنند.

			فرض کنید در زمان $t=0$ مشاهده $N$ اتم رادیواکتیو مستقل را آغاز می‌کنیم که هر کدام با $x^i_t$ برای $i = 1, 2, \ldots, N$ نمایش داده می‌شوند. اگر $X_t = \sum_{i=1}^N x^i_t$ قرار دهیم، آنگاه $\{X_t\}$ نیز یک فرآیند تصادفی است. نشان دهید که برای $t \leq 1/\lambda$ (یعنی $t$ در مقایسه با $1/\lambda$ ناچیز باشد) و $N$ به اندازه کافی بزرگ، فرآیند $\{X_t\}$ را می‌توان با تقریب بسیار خوبی توسط یک فرآیند پواسن $Y(t)$ با پارامتر $\lambda N t$ مدل کرد.

		\end{problem}


		\begin{problem}
			فرض کنید $\mathbb{R}$ یک فرآیند زاد و مرگ در زمان پیوسته باشد که در آن $\lambda_n = \lambda > 0$ برای $n \geq 0$، $\mu_0 = 0$ و $\mu_n > 0$ برای $n \geq 1$. همچنین فرض کنید $\pi = \sum_n \pi_n < \infty$ که در آن $\pi_n = \lambda^n / (\mu_1 \mu_2 \ldots \mu_n)$ به طوری که $\pi_i / \pi$ توزیع پایدار فرآیند است. اگر حالت اولیه یک متغیر تصادفی با توزیع برابر توزیع مانای فرآیند باشد، ثابت کنید که تعداد مرگ‌ها در بازه $[0, t]$ دارای توزیع پواسن با پارامتر $\lambda t$ است.

			\noindent
			\textbf{راهنمایی:} اگر $a_k(t)$ احتمال اینکه تعداد مرگ‌ها تا زمان $t$ برابر $k$ باشد را نشان دهد، معادله دیفرانسیل زیر را استخراج کنید:
			\[
			a_k'(t) = -\lambda a_k(t) + \lambda a_{k-1}(t), \quad k = 1, 2, \ldots
			\]
		\end{problem}
		\begin{problem}
تعریف زیر یک مفهوم از فرآیند پواسن چندمتغیره در دو بعد را ارائه می‌دهد. فرض کنید 
$(X(t), Y(t))$
به صورت 
$X(t) = \alpha(t) + \gamma(t)$
و 
$Y(t) = \beta(t) + \gamma(t)$
تعریف شود، که در آن 
$\alpha(t)$، $\beta(t)$
و 
$\gamma(t)$
سه فرآیند پواسن مستقل با پارامترهای 
$\lambda_1$، $\lambda_2$
و 
$\lambda_3$
هستند. تابع مولد توزیع 
$(X(t), Y(t))$
را بیابید.

% \noindent
% \textbf{پاسخ:}
% \[
% \sum \Pr\{X(t) = i, Y(t) = j\} x^i y^j
% \]
% \[
% = \exp\{t(\lambda_1 x + \lambda_2 y + \lambda_3 xy - \lambda_1 - \lambda_2 - \lambda_3)\}.
% \]
\end{problem}

\begin{problem}
فرض کنید زنجیره مارکوف متشکل از حالت‌های 0، 1، 2، 3 ماتریس احتمال انتقال زیر را داشته باشد:
\begin{equation*}
P=
\begin{bmatrix}
0 & 0 & \frac{1}{2} & \frac{1}{2} \\
1 & 0 & 0 & 0\\
0 & 1 & 0 & 0 \\
0 & 1 & 0 & 0
\end{bmatrix}
\end{equation*}
تعیین کنید کدام حالت‌ها گذرا و کدام بازگشتی هستند.
\end{problem}

\begin{problem}
اگر زنجیره مارکوف تقلیل‌ناپذیر و بازگشتی باشد، آنگاه برای هر حالت اولیه:
\begin{equation*}
\pi_j=\frac{1}{m_j}
\end{equation*}
\end{problem}

\begin{problem}
فرض کنید $\|P_{ij}\|$ ماتریس احتمال انتقال زنجیره مارکوف باشد و $\{\pi_j\}$ توزیع مانای فرآیند را نشان دهد. همچنین $\|P^{(m)}_{ij}\|$ ماتریس احتمال انتقال $m$-گام را نشان می‌دهد. اگر $\varphi(x)$ یک تابع مقعر روی $x \geq 0$ باشد و تعریف کنیم:
\[
E_m = \sum_{j=1}^N \pi_j \varphi(P^{(m)}_{ji}) \quad \text{با $l$ ثابت}
\]
ثابت کنید که $E_m$ تابعی نزولی نباشد از $m$ است، یعنی برای همه $m \geq 1$ داشته باشیم $E_{m+1} \geq E_m$.

\noindent
\textbf{راهنمایی:} از نامساوی جنسن استفاده کنید.
\end{problem}

\begin{problem}
ک بازیکن وجود دارند که بازیکن i ام مقدار $v_i > 0$ دارد ($i = 1,...,k$). در هر دوره، دو بازیکن بازی می‌کنند در حالی که $k-2$ بازیکن دیگر در یک صف مرتب منتظر می‌مانند. بازنده بازی به انتهای صف می‌پیوندد و برنده سپس با بازیکنی که در ابتدای صف است بازی جدیدی انجام می‌دهد. هرگاه i و j بازی کنند، i با احتمال $\frac{v_i}{v_i+v_j}$ برنده می‌شود.
\end{problem}

\begin{problem}
مشتریان می‌توانند توسط هر یک از سه سرویس‌دهنده سرویس دریافت کنند، که زمان‌های سرویس سرویس‌دهنده i دارای توزیع نمایی با نرخ $\mu_i$ است ($i = 1, 2, 3$). هرگاه یک سرویس‌دهنده آزاد شود، مشتری که بیشترین زمان انتظار را داشته است سرویس خود را با آن سرویس‌دهنده آغاز می‌کند.
\begin{enumerate}[label=(\alph*)]
\item اگر شما وارد شوید و هر سه سرویس‌دهنده مشغول و هیچ مشتری در انتظار نباشد، زمان مورد انتظار تا ترک سیستم را بیابید.
\item اگر شما وارد شوید و هر سه سرویس‌دهنده مشغول و یک مشتری در انتظار باشد، زمان مورد انتظار تا ترک سیستم را بیابید.
\end{enumerate}
\end{problem}

\begin{problem}
در یک سال با آب‌وهوای خوب، تعداد طوفان‌ها دارای توزیع پواسن با میانگین 1 است؛ در یک سال بد، این توزیع با میانگین 3 است. فرض کنید وضعیت آب‌وهوای هر سال فقط به وضعیت سال قبل بستگی دارد. یک سال خوب به احتمال مساوی با یک سال خوب یا بد دنبال می‌شود، و یک سال بد با احتمال دو برابر بیشتر با یک سال بد نسبت به یک سال خوب دنبال می‌شود. فرض کنید سال گذشته (سال 0) یک سال خوب بوده است.
\begin{enumerate}[label=(\alph*)]
\item تعداد مورد انتظار کل طوفان‌ها در دو سال آینده (یعنی سال‌های 1 و 2) را بیابید.
\item احتمال عدم وجود طوفان در سال 3 را بیابید.
\item میانگین تعداد طوفان‌ها در سال در بلندمدت را بیابید.
\item نسبت سال‌هایی که طوفان ندارند را بیابید.
\end{enumerate}
\end{problem}

\end{document}
